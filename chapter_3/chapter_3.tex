\pagestyle{fancy}
\headheight 20pt
\lhead{Ph.D. Thesis --- R. Woods}
\rhead{McMaster - Physics \& Astronomy}
\chead{}
\lfoot{}
\cfoot{\thepage}
\rfoot{}
\renewcommand{\headrulewidth}{0.1pt}
\renewcommand{\footrulewidth}{0.1pt}

\chapter{The Numerical Method}
\label{chap:method}
\thispagestyle{fancy}

In the absence of absorbing material, the problem of radiative transfer reduces down to that of gravity. As such, the tree-algorithm for calculating gravity can be used \citep{barnesHut86}.

\begin{itemize}
\item A tree can be used to partition space.
\item Each level of the tree holds finer partitions of the volume. See figure \ref{fig:tree}
\item Each node of the tree contains accumulated information about the tree below it (total mass, etc.).
\item In order to calculate gravity on a particular leaf (bucket), you can interact with the moment of another cell (\ref{eq:gravitymoment}).
\item To decide what level of the tree to interact with, you can define an opening angle/radius, $\theta$. If a cell is smaller than this opening angle (the distribution of matter inside the cell is contained within a small enough angle on the sky), the entire cell can be used in the force calculation. If not, you must consider the child nodes separately. See equation \ref{eq:openingangle}.
\item On average, the number of interactions a each particle will have is $\log{N}$, where N is the total number of particles. Thus, the force calculation for the whole simulation scales as $N\log{N}$. Note that lowering $\theta$ shifts the number of calculations that are approximated by large cells to smaller cells, and thus if $\theta$ is very small, the code approached scaling of order $N^2$.
\item In the case of radiation, the math is very similar (See eq \ref{eq:radiationmoments}). However, since radiation does not cancel like forces, the dipole moment does not disappear and a rougher approximation is possible (wording wrong, fix this).
\item In this case, the interaction scales as $N_{\mbox{sink}}\log{N_{\mbox{source}}}$. However, assuming the full tree is still used, the tree-build still scales as $N\log{N}$.
\end{itemize}

\section{Tree Data Structures}



\subsection{kd-Tree}

\section{Building a Radiation Tree}

\subsection{Criteria for Opening Cells}

\subsection{Accumulating Cell Properties}

\section{The Simple Case - No Absorption}

\subsection{Exchanging Radiation}

\section{Adding Absorption}

\subsection{Making Use of the Tree}

\section{Refinement}

\subsection{Criteria to Refine}

\section{Resolving the Sending and Receiving Cells}
