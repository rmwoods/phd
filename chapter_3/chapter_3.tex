\pagestyle{fancy}
\headheight 20pt
\lhead{Ph.D. Thesis --- R. Woods}
\rhead{McMaster - Physics \& Astronomy}
\chead{}
\lfoot{}
\cfoot{\thepage}
\rfoot{}
\renewcommand{\headrulewidth}{0.1pt}
\renewcommand{\footrulewidth}{0.1pt}

\chapter{The Numerical Method}
\label{chap:method}
\thispagestyle{fancy}

In the absense of absorbing material, the problem of radiative transfer reduces down to that of gravity. As such, the tree-algorithm for calculating gravity can be used \citep{barnesHut86}.

\begin{itemize}
\item A tree can be used to partition spacec
\item Each level of the tree holds finer partitions of the volume
\item Each node of the tree contains accumulated information about the tree below it (total mass, etc.).
\item In order to calculate gravity on a particular leaf (bucket)...
\end{itemize}

\section{Tree Data Structures}

\subsection{kd-Tree}

\section{Building a Radiation Tree}

\subsection{Criteria for Opening Cells}

\subsection{Accumulating Cell Properties}

\section{The Simple Case - No Absorption}

\subsection{Exchanging Radiation}

\section{Adding Absorption}

\subsection{Making Use of the Tree}

\section{Refinement}

\subsection{Criteria to Refine}

\section{Resolving the Sending and Receiving Cells}
