\pagestyle{fancy}
\headheight 20pt
\lhead{Ph.D. Thesis --- R. Woods}
\rhead{McMaster - Physics \& Astronomy}
\chead{}
\lfoot{}
\cfoot{\thepage}
\rfoot{}
\renewcommand{\headrulewidth}{0.1pt}
\renewcommand{\footrulewidth}{0.1pt}

\chapter{Radiative Transfer}
\label{chap:radtransfer}

\thispagestyle{fancy}

This chapter will contain an overview of current radiative transfer methods and where we stand.

\section{The Radiative Transfer Problem}
\label{sec:rtformulation}


\begin{equation}
\label{eq:emissioncoef}
j = \frac{dE}{dV d\Omega dt}
\end{equation}

\begin{equation}
\label{eq:intensity}
dI = j ds
\end{equation}

\begin{equation}
\label{eq:absorption}
dI = -\alpha I ds = -n \sigma I ds = -\rho \kappa I ds
\end{equation}

\begin{equation}
\label{eq:radtransfer}
\frac{dI}{ds} = -\alpha I +j
\end{equation}

\begin{equation}
\label{eq:rademission}
I(s) = I(s_0) + \int_{s_0}^{s} j(s') ds'
\end{equation}

\begin{equation}
\label{eq:radabsorption}
I(s) = I(s_0)\exp{\left[-\int_{s_0}^{s} \alpha(s') ds'\right]} = I(s_0)\exp{\left[ -\tau (s) \right]}
\end{equation}

\begin{equation}
\label{eq:opticaldepth}
\tau(s) = \int_{s_0}^{s} \alpha (s') ds' = \int_{s_0}^{s} \rho (s') \kappa(s') ds'
\end{equation}

\section{Current Methods}
\label{sec:currentmethods}

The current set of computational methods can be classified as follows:

\begin{itemize}
\item Very accurate, expensive methods - monte carlo, ray tracing
\item Methods accurate in specific scenarios, e.g. FLD for optically thick, ``str\"omgren method'' from Dale
\item \emph{Very} rough approximations, e.g. method that only looks at absorption near sink and source.
\end{itemize}

As will be seen, there is currently an opening in the market for something in the middle - Decent accuracy at a low cost.

%\subsection{Numerical Strategy for Solving RT}
%\label{}

\begin{itemize}
\item RT is a function of seven variables - $x, y, z, \theta, \phi, t, \nu$.
\item Simple division of each variable into 100 bins means storage of over 10$^{14}$ elements, or roughly 1 Petabyte of data if each piece of information stored was ten bytes. First problem - large memory cost.
 \begin{itemize}
 \item Note that 100 elements is minimal in many cases due to sharply peaked functions, e.g. $\nu$.
 \item 100x100 for angle only gives angular resolution of roughly 2.7$^{\circ}$ [not quite right - check this (steinacker 09 book)] 
 \end{itemize}
\item RT equation is an integro-differential equation - difficult to use any sort of common solver.
\end{itemize}

Strategies include monte-carlo, ray tracing, grid-based solvers, and moment methods, each with certain advantages and disadvantages. [Following sections are first pass. Brief overview of advantages and disadvantages. Second pass will give a basic intro into how the algorithm works so that it's clear why the methods have the advantages/disadvantages they do. Can also compare/contrast our method with current ones better.

\subsection{Monte-Carlo Solvers}
\label{montecarlo}

In monte-carlo methods, a photon is carefully tracked through a domain, following scattering, absorption, and re-emission. See [from steinacker 09] \citet{wolf03,woodEt2004,ercolanoEt2005,jonsson06,pinteEt06}. [ADD method basics]


\begin{itemize}
\item Advantages - can treat complicated spatial distributions, arbitrary scattering functions, and polarization.
\item Disadvantages - Very high or low optical depths hard (why?), re-emission in all directions over many events hard (why?), no global error control
\end{itemize}


\subsection{Ray Tracing}
\label{sec:raytracing}

Ray tracing has the ability to treat arbitrary density distributions, and can use general solvers for ODEs. Fairly accurate and fairly expensive.

\begin{itemize}
\item Often combined with monte-carlo methods
\item Provides quite accurate results.
\item Provides global error control.
\item Can cause step-size limitations in order to consistently transfer photons.
\item Becomes impractical at high optical depth and can require complicated solvers.
\end{itemize}


\subsection{Grid-Based Methods}
\label{sec:gridmethods}

Can use simple solvers (finite differencing or short characteristics) and gives good error control.

\begin{itemize}
\item Grid must be adaptable to be practical, though good refinement criteria is unclear
\item Interpolation between grids can be costly and give interpolation errors.
\item Numerical diffusion typically not taken into account.
\end{itemize}


\subsection{Moment Methods}
\label{sec:momentmethods}

e.g. FLD or M1 moment codes (skinner + ostriker, others). Fairly accurate, faster than ray tracing, in specific regimes.

\begin{itemize}
\item Well suited for optically thick regime (can do free streaming)
\item Struggles with peaked radiation in optical depths of order 1.
\end{itemize}


