\pagestyle{fancy}
\headheight 20pt
\lhead{Ph.D. Thesis --- R. Woods }
\rhead{McMaster - Physics \& Astronomy}
\chead{}
\lfoot{}
\cfoot{\thepage}
\rfoot{}
\renewcommand{\headrulewidth}{0.1pt}
\renewcommand{\footrulewidth}{0.1pt}

\chapter{Applications to Galaxy Formation (Future Work?)}
\label{chap:galaxyformation}
\thispagestyle{fancy}

In this chapter, we discuss current and planned applications of the algorithm.

\section{Galactic UV Fields}
\label{sec:galacticuvfields}

Currently, there is very little work in computational astrophysics that models the UV fields in and around galaxies. While many models have been created from the observational side [references], due to large computational cost, simulations have left this area largely unexplored or explored only at high redshift [references].

We have used [will be using] the above algorithm to re-run the MUGS2 comparison project [cite Ben].

\begin{itemize}
\item What effect does radiation have on the galaxy in MUGS2 (as a function of redshift, all the way to redshift 0)?
 \begin{itemize}
 \item What effect does UV have on the ISM? (sam?)
 \item What effect does UV have on satellite galaxies, gas properties, SFR - shut down mechanism?
 \item What is the typical escape fraction of UV in galaxies (isolated disk from Sam, consistent with results from MUGS? Compare to Kannan et al. escape fractions)
 \item Compare to observations of UV fields
 \end{itemize}
\end{itemize}

Future work of this algorithm is quite broad; the flexibility allows application to a wide range of problems. The following are immediately planned projects, and following that is a short list of unplanned but possibly interesting projects.

\begin{itemize}
\item Look at H2 formation and destruction using Ly-Werner bands of radiation (Charlotte). How does cloud shielding depend on density? Compare to OWLS/EAGLE w/ TRAPHIC (Rahmati+ 13ab) (this is HI shielding and ntot H2 shielding... not great comparison?).
\item Considering the properties of the ISM and molecular clouds (Samantha, Sijing). [Author] suggests 4 radiation bands are needed to sufficiently find ISM properties. Using these 4 bands, calculate effect of radiation on ISM. How do gas properties effect SFR and vice versa? [ask Sam for more info here]
\item Potential to look at the effect of radiation processing. How does processing radiation (UV re-emitted as IR) effect the gas properties? How important of an effect is it in determining SFR? This project depends on a successful implementation of a self consistent gas source function. Can we do this in a stable way with respect to the cooling code?
\item An obvious application is cosmic re-ionization, but this has already been done a bit since it does not require running to low redshift. That said, can we do it better/cheaper? Do our results agree? 
\end{itemize}

%Future Work
%Cosmology
%Reionization of the universe
%·         not original goal, but is possible
%
% 
%Running simulations with radiative transfer for systems with low red shift
%·         most radiative transfer simulations are very computationally expensive
%·         Looking at galactic UV fields
%o   from the gas inside of the galaxy
%o   around the galaxy
%o   and the satellites around it
%o   has potential link to observations
%·         Lyman-Werner radiation and molecular hydrogen formation/destruction
%
% 
%Properties of molecular clouds (Properties of the ISM)
%·         smaller scale
%·         how does the clod get changes after star formation
%
%
%Radiations for gas
%·         stars radiation is fairly straight forward, but adding the radiation from the gas would not be that computationally expensive
%·         line emission from gas
%·         being able to treat all of the gas particles as emitting gas
%·         not sure if can be done in a self-consistent way (does the radiation correspond to the amount that it cools be and is it stable?)