\pagestyle{fancy}
\headheight 20pt
\lhead{Ph.D. Thesis --- R. Woods}
\rhead{McMaster - Physics \& Astronomy}
\chead{}
\lfoot{}
\cfoot{\thepage}
\rfoot{}
\renewcommand{\headrulewidth}{0.1pt}
\renewcommand{\footrulewidth}{0.1pt}

\chapter{Introduction}
\label{chap:intro} 
\thispagestyle{fancy} 

First part - what is this document? What will be in it? What is the main thing I will be telling you about? [Move this after intro to astro stuff? I think so...].


[abelnormanmadau has good cosmo-intro w/ references.]

\section{Astrophysics and Radiation}
\label{sec:astroandrad}

It doesn't take much to convince a physicist of the importance of photons - astrophysical object speak in photons. As astronomers, we receive all of our information in the universe through photons. In order to understand the objects we observe, we must understand photons; how are they created? How are they removed? What processes can alter a photon?

\begin{itemize}
\item Introduce stellar systems -  radiation can provide pressure support, can cause ionization, dissociation, can heat the gas. Because gas properties tied to star formation rate, and stars are one of the primary targets of observation, it's important to know what's going on.
\item Introduce cosmology (how in depth here?). Cosmology depends on knowing photon history and what cosmological factors affect photons. Introduce cosmic reionization, Universe expansion, SZ?, lensing?, CMB?
\item Need to ability to track photoionization, photoheating across huge amounts of sources and their environments. Astrophysical radiation has both short and large scale effects that can't be ignored.
\end{itemize}

%Introduction to the basics of cosmology
%·         properties in the context of computational radiative transfer
%
%·         important evends like re-ionization of the universe
%
%·         ionization fronts and large numbers of sources
%
%·         with a focus on the physics side of things
%
% 
%Astro relies on photons - language of the universe
%·         only way the universe can be observe
%
%·         needs to be understood to get an idea of how things are interacting
%
% 
%Look at cosmology and stellar systems where these things are really important. 
%·         In order to understand why the algorithm needs to function the way it does, need to understand cosmology to have relevant applications
%
%·         Need to develop a what to represent approximations to deal with the large length scales and large number of sources.


\section{Overview}

Chapter \ref{chap:radtransfer} will go over the background of radiative transfer, etc.
